\documentclass[a4paper, 12pt]{article}

\usepackage{exercise-sheet}
\usepackage{global-macros}
\usepackage{biblatex}
\usepackage{csquotes}
\usepackage{hyperref}
\usepackage{graphicx}
\graphicspath{{../fig}}
\usepackage{float}

\hfuzz=16pt

\addbibresource{../bib/references.bib}

\title{Modelos de classificação em previsão de renda} 
\author{Professor: Rodrigo Targino \\
        Aluno: Caio Lins}
\date{\today}

\begin{document}

\maketitle

\tableofcontents

\section{Recap do conjunto de dados}

% Relembrar qual é o dataset, dizer o que foi feito para lidar com
% dados faltantes, falar sobre divisão entre teste e treino

\section{Aplicação dos modelos}

% Discussão sobre datasets desbalanceados: O que fazer para resolver
% undersampling, oversampling, pesos, etc. Vantagens e desvantagens de
% cada um

% Fazer uma subseção por modelo. Discutir performance, quais variáveis o
% modelo indica serem as mais relevantes, abordar as características de
% cada modelo, falar sobre coisas que foram utilizadas e que não foram
% vistas em sala. Colocar matrizes de confusão, plots de curva ROC, as
% árvores construídas. Fazer coisas bonitinhas.

% Modelos utilizados:
% Perceptron
% Regressão Logística
% Métodos baseados em árvores
    % Uma árvore
    % bagging
    % random forest
    % boosting
% SVMs

\section{Conclusão}

\end{document}

