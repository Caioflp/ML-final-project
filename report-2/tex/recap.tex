\section{\emph{Recap} do conjunto de dados}

Relembrando o que foi apresentado no relatório anterior, estamos trabalhando com o \emph{``Census Income Data Set''}~\cite{uci}.
Nossa tarefa é classificar cada indivíduo que participou do censo com relação à sua renda: se ele ganha mais que \$ 50 000 por ano, ou não.
À nossa disposição, temos uma série de variáveis de caráter socioeconômico, categóricas e quantitativas, dentre as quais gostaríamos de identificar aquelas que melhor explicam a variável resposta.
Para tanto, treinamos, no nosso conjunto de dados, modelos de classificação que possuem uma boa explicabilidade, ou seja, que não são modelos do tipo ``caixa preta''.
Foram eles, em ordem crescente de complexidade: \emph{perceptron}, regressão logística, modelos baseados em árvores (uma árvore, \emph{bagging} e \emph{random forest}) e \emph{support vector machines (SVM)}.

\subsection{\emph{Old problems and new}}

No último relatório foi apontado que o conjunto de dados apresentava algumas lacunas, em três \emph{features} diferentes.
Apesar de, inicialmente, termos considerado utilizar técnicas estatísticas para preencher os dados, como, por exemplo, o algoritmo EM, por uma questão de (falta de) recursos humanos não foi possível implementar essa ideia.
A solução utilizada foi a mais simples: descartamos as amostras que possuíam alguma \emph{feature} faltante.
Como elas eram poucas (\( \sim 2800 \), contra as \( \sim 48 000 \) totais), isso não representou uma perda muito significativa.

Entretanto, uma questão não considerada anteriormente se apresentou quando começamos as treinar modelos: o \textbf{desbalanceamento do \emph{dataset}}.
A proporção entre instâncias positivas (ganham mais que 50K) e negativas é de, aproximadamente, 1:3, ou seja, \( \sim 25 \% \) dos indivíduos tem renda elevada.
Apesar de não ser exageradamente expressiva, essa assimetria faz com que haja uma tendência para classificar uma determinada instância como negativa.

Ressaltamos que isso não necessariamente é um problema, pois tudo depende de qual é o objetivo da análise que está sendo conduzida.
Se a proporção entre classes fora do \emph{dataset} é a mesma de dentro \emph{dataset} e, por algum motivo, deseja-se apenas obter o modelo com melhor \emph{score}
\footnote{Proporção de instâncias classificadas corretamente.},
não importa muito se há um forte viés para alguma das classes.
Entretanto, raramente esse é o caso em aplicações reais de \emph{machine learning}.
No nosso problema, por exemplo, não é útil que todos os coeficientes da regressão logística sejam muito baixos, pois isso diz pouco sobre a relevância de cada um para explicar a variável resposta.
Portanto, foi necessário utilizar uma estratégia para contrabalancear o viés no conjunto de dados.
