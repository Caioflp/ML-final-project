\section{Conclusão}

Neste trabalho, nos propusemos a utilizar modelos de classificação para estudar o conjunto de dados \cite{uci}, com objetivo de prever/explicar a variável resposta, relativa à renda dos indivíduos.

Nesse sentido, na primeira parte realizamos uma análise exploratória, idenficando variáveis de interesse e possíveis problemas que deveríamos contornar.
Em primeiro lugar, nos deparamos com dados faltantes.
A solução utilizada para esse problema foi a de simplesmente desconsiderar os indivíduos que possuíssem alguma \emph{feature} latente.
Em segundo lugar, percebemos que o desbalanceamento do \emph{dataset} poderia criar um viés indesejado em favor da classe negativa.
Para contrabalancear essa tendência, utilizamos um mecanismo de pesos para cada classe, que demonstrou ser efetivo em quase todos os modelos analisados.

Ao treinarmos os classificadores, vimos que a estratégia de validação cruzada utilizada foi efetiva, pois as perdas nos conjuntos de treino e teste ficaram muito próximas.
Isto é, não houve overfitting significativo.
Também vimos que as \emph{features} que mais foram classificadas como relevantes foram: \verb|capital_gain|, o grau de educação, a idade e as relações conjugais de cada indivíduo.
A \emph{feature} \verb|capital_gain| é a única cujo significado não é inteiramente conhecido, por não ser informado na página do \emph{dataset}.
Por ser diferente de \verb|target|, podemos inferir que se trata de ganhos monetários além de salários.
